\documentclass[11pt,a4paper]{article}
\usepackage[utf8]{inputenc}
\usepackage[czech]{babel}
\usepackage[T1]{fontenc}
\usepackage{amsmath}
\usepackage{amsfonts}
\usepackage{listings}
\usepackage{color}
\usepackage{epstopdf}
\usepackage{svg}
\definecolor{lightgray}{rgb}{.9,.9,.9}
\definecolor{darkgray}{rgb}{.4,.4,.4}
\definecolor{purple}{rgb}{0.65, 0.12, 0.82}
\usepackage{url}
\usepackage{amssymb}
\usepackage{graphicx}
\bibliographystyle{documentation}
\usepackage[left=2cm,right=2cm,top=2.5cm,bottom=2cm]{geometry}
\author{Tomáš Kohout}
\begin{document}



\lstdefinelanguage{code}{
  keywords={},
  keywordstyle=\color{blue}\bfseries,
  ndkeywords={},
  ndkeywordstyle=\color{darkgray}\bfseries,
  identifierstyle=\color{black},
  sensitive=false,
  comment=[l]{//},
  morecomment=[s]{/*}{*/},
  commentstyle=\color{purple}\ttfamily,
  stringstyle=\color{red}\ttfamily,
  morestring=[b]',
  morestring=[b]"
}

\lstset{
   language=code,
   backgroundcolor=\color{lightgray},
   extendedchars=true,
   basicstyle=\footnotesize\ttfamily,
   showstringspaces=false,
   showspaces=false,
   numbers=left,
   numberstyle=\footnotesize,
   numbersep=9pt,
   tabsize=2,
   breaklines=true,
   showtabs=false,
   captionpos=b
}

\DeclareGraphicsExtensions{.pdf,.png,.jpg}

\newcommand{\slideRef}[1]{\textit{(IMS slide #1)}}
\newcommand{\code}[1]{\texttt{#1}}

	\begin{titlepage}

		\begin{center}

			\textsc{
				\Huge
					Vysoké učení technické v~Brně\\
				\huge
					Fakulta informačních technologií
			}\\

			\vspace{\stretch{0.320}}

			\LARGE
					IMS - Modelování a simulace\\
					4. Doprava zboží nebo osob\\~\\
			\Huge{}
					Dokumentace
			\vspace{\stretch{0.650}}

		\end{center}

		{\Large
			29. listopadu 2017
			\hfill
			Tomáš Blažek, Tomáš Kohout
		}

	\end{titlepage}

	\tableofcontents

	\pagebreak


	\section{Úvod a motivaci}
		Tato práce si klade za cíl zjistit, zda dosavadní frekvence převážení cestujících přívozem na trase Podbaba-Podhoří je v
    souladu s množstvím přepravovaných osob nebo zda je možné tuto frekvenci snížit. Na základě vytvořeného modelu
    budou provedeny experimenty s různými časovými úseky a různým množstvím cestujících.

    Tato práce si klade za cíl zjistit, zda je dosavadní frekvence převážení cestujících přívozem na trase Podbaba-Podhoří
    v souladu s množstvím převážených cestujících. Pro tento účel byl vytvořen model (\cite{SLAJD}, 7. slide)

		\subsection{Konzultace a zdroje}
			Autory projektu jsou Tomáš Blažek a Tomáš Kohout. Pro vytvoření projektu jsou použity veřejně dostupné informace ze stránek Dopravního podnku hl. m. Prahy, především pak jízdní řád a informace o celkovém počtu přepravených osob. Abstraktní model je vytvořen na základě jizdního řádu a pozorování .

		\subsection{Ověřování validity modelu}
    Model byl ověřován experimenty prováděné pomocí modelu. Získané výsledky byly následně porovná-
    vány s reálnými údaji získanými na základě pozorování obsluhy pasažérů \cite{DELKA_CESTY} a také pomocí získaných dat. \cite{ROCENKY}


	\section{Rozbor tématu a použitých metod/technologií}
  Pro modelování a následnou simulaci přívozu je nezbytné znát jeho reálné chování.
  Přívoz z Podbaby do Podhoří se zabývá výlučně přepravou cestujících z jednoho
  břehu na druhý. Tuto informaci jsme získali pozorováním z následujícího videa \cite{DELKA_CESTY}.
  Z toho samého snímku je patrná i doba cesty, která je v tomto konkrétním případě 1 minuta a 28 sekundy
  Na stejném videu jsme také vypozorovali průměrnou dobu vystupování. Ta se pohybuje
  okolo 5 sekund v případě staršího typu lodě, který je vidět na videu. V současné době je ovšem používána loď zakoupená
  v roce 2016 \cite{LOD}, která má palubu v rovině s molem a vyšší kapacitu. Kapacita se z původních 12 míst zvýšila na
  40. Dále bylo třeba zjistit celkový počet jízd přívozu za jeden pracovní den. Tuto informaci
  jsme zjistili z jízdního řádu \cite{PID}, který je platný od 11.12.2016.
  V ročenkách Technické správy komunikací hl. m. Prahy jsme získali data potřebná pro
  správné generování cestujících \cite{ROCENKY}. Tato hodnota se pohybuje v rozsahu od 190 000
  až po 350 000 převezených cestujících ročně.

  \subsection{Použíté postupy pro vytváření modelu}
  Pro vytvoření byl použitý jazyk C++, neboť v tomto jazyku ja napsaná knihovna SIMLIB \cite{SIMLIB},
  která je vhodná pro simulaci daného zadání. V modelu jsou použity algoritmy použité
  128  - 206


	\section{Závěr}

	\section{Zdroje}
		\bibliography{documentation}


\end{document}
