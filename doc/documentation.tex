\documentclass[11pt,a4paper]{article}
\usepackage[utf8]{inputenc}
\usepackage[czech]{babel}
\usepackage[T1]{fontenc}
\usepackage{amsmath}
\usepackage{amsfonts}
\usepackage{listings}
\usepackage{color}
\usepackage{epstopdf}
\usepackage{svg}
\usepackage{url}
\usepackage{amssymb}
\usepackage{graphicx}
\bibliographystyle{documentation}
\usepackage[left=2cm,right=2cm,top=2.5cm,bottom=2cm]{geometry}
\author{Tomáš Kohout, Tomáš Blažek}
\begin{document}


\DeclareGraphicsExtensions{.pdf,.png,.jpg}

\newcommand{\slideRef}[1]{\textit{(IMS slide #1)}}
\newcommand{\code}[1]{\texttt{#1}}

	\begin{titlepage}

		\begin{center}

			\textsc{
				\Huge
					Vysoké učení technické v~Brně\\
				\huge
					Fakulta informačních technologií
			}\\

			\vspace{\stretch{0.320}}

			\LARGE
					IMS - Modelování a simulace\\
					4. Doprava zboží nebo osob\\~\\
			\Huge{}
					Dokumentace
			\vspace{\stretch{0.650}}

		\end{center}

		{\Large
			29. listopadu 2017
			\hfill
			Tomáš Blažek, Tomáš Kohout
		}

	\end{titlepage}

	\tableofcontents

	\pagebreak


	\section{Úvod a motivace}
	Tato práce si klade za cíl zjistit, zda dosavadní frekvence převážení cestujících přívozem na trase Podbaba-Podhoří je
v~souladu s~množstvím přepravovaných osob nebo zda je možné tuto frekvenci snížit. Na základě vytvořeného modelu  (\cite{SLAJD}, slide 7)
  budou provedeny experimenty s~různými časovými úseky a různým množstvím cestujících.

  Tato práce si klade za cíl zjistit, zda je dosavadní frekvence převážení cestujících přívozem na trase Podbaba-Podhoří
  v~souladu s~množstvím převážených cestujících. Pro tento účel byl vytvořen model (\cite{SLAJD}, 7. slide)

	\subsection{Konzultace a zdroje}
	Autory projektu jsou Tomáš Blažek a Tomáš Kohout. Pro vytvoření projektu jsou použity veřejně dostupné informace ze stránek Dopravního podnku hl. m. Prahy, především pak jízdní řád a informace o~celkovém počtu přepravených osob. Abstraktní model (\cite{SLAJD}, slide 40-43) je vytvořen na základě jizdního řádu a pozorování .

	\subsection{Ověřování validity modelu}
  Model byl ověřován experimenty prováděné pomocí modelu. Získané výsledky byly následně porovná-
  vány s~reálnými údaji získanými na základě pozorování obsluhy pasažérů \cite{DELKA_CESTY} a také pomocí získaných dat \cite{ROCENKY}.

	\section{Rozbor tématu a použitých metod/technologií}
  Pro modelování a následnou simulaci přívozu je nezbytné znát jeho reálné chování.
  Přívoz z~Podbaby do Podhoří se zabývá výlučně přepravou cestujících z~jednoho
  břehu na druhý. Tuto informaci jsme získali pozorováním z~následujícího videa \cite{DELKA_CESTY}.
  Z~toho samého snímku je patrná i doba cesty, která je v~tomto konkrétním případě 1 minuta a 28 sekundy
  Na stejném videu jsme také vypozorovali průměrnou dobu vystupování. Ta se pohybuje
  okolo 5 sekund v~případě staršího typu lodě, který je vidět na videu. V~současné době je ovšem používána loď zakoupená
  v~roce 2016 \cite{LOD}, která má palubu v~rovině s~molem a vyšší kapacitu. Kapacita se z~původních 12 míst zvýšila na
  40. Dále bylo nutné zjistit celkový počet jízd přívozu za jeden pracovní den. Tuto informaci
  jsme zjistili z~jízdního řádu \cite{PID}, který je platný od 11.12.2016.
  V~ročenkách Technické správy komunikací hl. m. Prahy jsme získali data potřebná pro
  správné generování cestujících \cite{ROCENKY}. Tato hodnota se pohybuje v~rozsahu od 190 000
  až po 350 000 převezených cestujících ročně. Předpokládáme, že v~letním období je
  zájem o~přepravu přívozem větší než v~zimě. Nejsou však k~dispozici žádné informace
o~počtu převezených pasažérů v~letním období, budeme tedy uvažovat, že v~letním období
  převeze přívoz denně přibližně 2 - 3 krát více osob než v~zimnímm období.

  \subsection{Použíté postupy pro vytváření modelu}
  Postup se skládá z~několika kroků. Nejprve bylo nutné dekomponovat úlohu na jednotlivé procesy,
  které se v~ní odehrávají. Petriho síť(\cite{SLAJD}, slide 123-142), byla vhodným kandidátem pro grafické znázornění
  jednotlých procesů, ze které se dále vycházelo při tvorbě simulačního modelu.

  \subsection{Původu použitých metod/technologií}
  Pro vytvoření simulačního modelu byl použitý jazyk C++, neboť v~tomto jazyce je napsaná knihovna SIMLIB \cite{SIMLIB},
  která je vhodná pro vytvoření simulačního modelu daného problému, protože obsahuje funkce, které vystihují
  chování systému hromadné obsluhy. V~modelu jsou použity algoritmy využívané v~příkladech na democvičení \cite{DEMOCVIKO}.

  \section{Koncepce modelu}
  Smyslem tohoto projektu je simulovat (\cite{SLAJD}, slide 8) provoz přívozu Pražské integrované dopravy
  z~Podbaby do Podhoří v~určitém časovém intervalu a sledovat zahlcenost nebo nezahlcenost kapacity přívozu,
  popřípadě délku front na obou březích a vytvořit tak ideální frekvenci jízd.

  Zanedbali jsme povodňové stavy a různé poruchy, protože pro simulaci (\cite{SLAJD}, slide 8)
  jsou marginální. Přívoz při prvním povodňovém stupni nevyjíždí a poruchy různého druhu se
  můžou vyskytnout, ale pro zjištění frekvence jízd nejsou podstatné.

  \subsection{Návrh konceptuálního modelu}
  Přívoz má dvě mola. Ke každému molu přicházejí cestující s~exponenciálním rozložením (\cite{SLAJD}, slide 91).
  Tito cestující čekají na uvolnění mola. Je-li molo k~dispozici, není vyčerpána kapacita lodě a nenastal čas odjezdu
  nastupují po jednom pasažérovi do lodě. Nastupování samo o~sobě zabere určitý čas. Pokud nastane čas odjezdu, tak právě
  nastupující cestující dokončí svou činnost a loď vypluje na cestu. Po určité době dojede k~druhému molu, kde nejprve vystoupí naložení cestující.
  Následně mohou nastupovat čekající cestující. Nastupování probíhá totožně jako na prvním molu.
  Po uplynutí určité doby a dokončeném nastupování pasažéra, který nastupoval v~době odjezdu odjede loď zpět
k~prvnímu molu, kde opět vyloží naložené cestující a celý proces se opakuje. Doba čekání na druhém břehu musí být
  dostatečná pro vystoupení a nastoupení plně obsazené lodě.

  \subsection{Forma konceptuálního modelu}

  Pro popis abstraktního modelu (\cite{SLAJD}, slide 40-43) byla vytvořena Petriho síť (\cite{SLAJD}, slide 123-142),
  ze získaných informací uvedených dříve v~textu. Tato síť je hierarchicky popsána na obrázcích \ref{pnTime}, \ref{pnWork} a \ref{pnWork2}.

	\begin{figure}[ht]
		\centering
		\includegraphics[scale=0.7]{pn-daytime.pdf}
		\caption{Petriho síť znázorňující režim odjezdů}
		\label{pnTime}
	\end{figure}

	Na obrázku \ref{pnTime} vidíme, že v~původním jízdní řádu je počet jízd udán číslem 68 a
	poté následuje 7 hodin pauza. Modré šipky indikují začátek generování nových pasažérů.
	Zelené šipky indikují ukončení generování pasažérů přes noc. Do zlata zbarevé stavy
	slouží k~vyprázdnění fronty čekajících pasažérů na odvoz, aby nečekali přes noc.
	Červená šipka uvolní molo Podhoří a umožní cestujícím nastupovat. Do červena zbarvený
	stav určuje, že po nastoupení posledního pasažéra se vydají naložení pasažéři na cestu.

	\begin{figure}[ht]
		\centering
		\includegraphics[scale=0.7]{pn-work.pdf}
		\caption{Petriho síť znázorňující nástup/výstup pasažérů a cestu ke druhému molu}
		\label{pnWork}
	\end{figure}

	\newpage
	Na obrázku \ref{pnWork} je znázorněno nastupování a vystupování cestujících. Pokud nastane čas
	odjezdu, tak se loď vydá na cestu a pluje ke druhému molu. Z~obrázku je zřejmé, že vždy dojde
	nejprve k~vystoupení cestujících a až posléze k~nastoupení nových pasažérů, přičemž k~odejzdu
	dojde až poté, co nastoupí poslední cestující. Celý abstraktní model používá pouze jednu kapacitu lodě.
	Na obrázku \ref{pnWork} a obrázku \ref{pnWork2} je znázorněn světle modrou barvou.
	Generování cestujících začíná vždy se začátkem nového dne a končí s~koncem dne.
	Na konci dne se všichni čekající cestující uvolní ze systému (\cite{SLAJD}, slide 7).

	\begin{figure}[ht]
		\centering
		\includegraphics[scale=0.7, width=\textwidth]{pn-work2.pdf}
		\caption{Petriho síť znázorňující nástup/výstup pasažérů a cestu k~prvnímu molu}
		\label{pnWork2}
	\end{figure}
	Zde vidíme, že po příjezdu do Podbaby se vytvoří nový proces, který čeká 4 minuty.
	Po uplynutí této doby se loď vydá zpět na cestu do Podhoří. V~tomto čase stihne vyložit
	cestujícící a naložit nové. Pokud je tato jízda poslední, tak je proces lodi ze systému (\cite{SLAJD}, slide 7) uvolněn.
	Stejně tak jsou uvolněni i pasažéři, kteří nemohou být naloženi z~důvodu noční fáze.

 \newpage
	\section{Architektura simulačního modelu}
	V~programu jsou implementovány 2 Facility(\cite{SLAJD}, slide 163). Ty reprezentují 2 mola, které umožňují nástup cestujících.
    Dále jeden prvek typu Store(\cite{SLAJD}, slide 163), který znázoruňuje palubu lodi s~určitou kapacitou. V~samotném simulačním
	modelu (\cite{SLAJD}, slide 9) jsou využívány 2 generátory pasažérů rozšiřující třídu Event((\cite{SLAJD}, slide 163)). Ty se starají o~generování
	pasažérů na obou březích. Jelikož v~reálném světě přicházejí cestují pouze přes den v~pracovní době přívozu,
	tak je generování pasažérů v~hodinách mimo pracovní dobu pozastaveno a na začátku pracovní doby opět spuštěno.
	Pasažéři jsou dvou typů, rozlišují se ve směru jízdy. Ti jsou implementovány jako třídy, které jsou rozšířené
	třídou Process(\cite{SLAJD}, slide 163). Při vygenerování pasažéra, je jeho první činnost zjistit, zda je možné využít molo k~nástupu
	do přistavené lodi nebo se postavit do čekací fronty. Celé toto řidí proces času, který znázorňuje
	dopravní řád přívozu a určuje kdy mohou pasažeři nastupovat či vystupovat.

	\section{Podstata simulačních experimentů a jejich průběh}

	Cílem experimentů je najít vhodný jízdní řád pro přívoz na trase Podbaba - Podhoří v~Praze.
	Ke zjištění této informace je vhodné vytvořit model (\cite{SLAJD}, slide 7),
	neboť provádění experimentů při ostrém provozu není možné z~důvodu platných smluvních
	podmínek Pražského dopravního podniku, které jsou uzavřeny s~každým cestujícím, který
	si zakoupí jízdenku.
	\subsection{Postup experimentů}
  V~simulačním modelu jsou možné úpravy u~následujících hodnot.
	\begin{enumerate}
		\item Určení rychlosti generování pasažérů.
		\item Počet jízd přívozu
		\item Kapacitu lodi
	\end{enumerate}

	Pro vytvoření nejvhodnější frekvence jízd jsme se rozhodli ovlivnit rychlost
	generování pasažérů, počet jízd za den a kapacitu lodi pro převoz pasažérů mimo
	sezónu.

	\subsection{Jednotlivé experimenty}

	\subsubsection{Experiment 1}
	Experiment číslo 1 je zaměřený na zkoumání obsazenosti lodi v~sezóně. Kapacita
	zůstává na 40 místech, doba experimentu je 91 dní \cite{WIKI:LETO} a generování pasažéra
	na každém břehu proběhne pomocí exponenciálního rozdělení se středem 76 sekund.
	Počet jízd je stanoven na současnou hodnotu, tedy 68 jízd kdy každých 15 minut
	vyjede loď z~Podhoří do Podbaby a zpět.	Z~výsledků je patrné, že loď nebyla využita
	na svůj maximální potenciál. Maximum nastoupených pasažérů je 25 a minimální počet je roven 0.
	Celkový počet pasažérů je 146 553. Průměrně strávili na cestě 555 sekund což je přibližně 9 minut.

 \subsubsection{Experiment 2}
  Experient číslo 2 je zaměřený na zkoumání obsazenosti lodi mimo sezónu. kapacita
	v~tomto experimentu zůstává jako v~předešlém na standardních 40 místech. Doba experimentu
	je 274 dní a generování pasažérů je upravené v~závislosti na počtu pasažérů v~sezóně,
	kteří se odečtou od celkového počtu pasažérů za celý rok. V~tomto případě bude generování
	pasažérů probíhat exponenciálním rozdělením se středem 678 sekund na každém břehu.
	Z~experimentu je patrné, že loď je využitá naprosto minimálně. Počet převážených pasažérů
	v~žádně jízdě nepřesáhl počet 9 a celkový počet vygenerovaných pasažérů je 49 883, což
	v~porovnání s~Experimentem 1 je třetina pasažérů, ale za trojnásobnou dobu.
	Z~tohoto experimentu je patrné, že frekvence jízd je absolutně nevyhovuje poptávce po převozu.

	\subsubsection{Experiment 3}
	Tento experiment reprezentuje nejvhodnější frekvenci jízd přívozu v~době sezóny. Počet dní sezóny je stanoven na 91.
	Počet jízd přívozu byl snížen na 41 jízd denně což přibližně odpovídá 25 minutovým výjezdům z~Podhoří do Podbaby.
	Kapacita lodi zůstala na 4O místech, generování pasažérů bylo stanoveno na běžnou hodnotu
	sezóny a to pomocí exponenciálního rozložení se středem 76 sekund pro vygenerování jednoho pasažéra
	na každém břehu.
	Maximální využití lodi bylo 37 pasažérů, přičemž žádný pasažér nemusel čekat na další jízdu
z~důvodu maximálního využití kapacity. Celkově bylo převezeno 146982 pasažérů.

	\subsubsection{Experiment 4}
	V~tomto experimentu jsme snížili počet jízd mimo sezónu na jednu za hodinu, tedy 17 jízd za den. Generování pasažérů
	bylo ponecháno na standardní mimosezónní úrovni, tedy exponenciální rozložení se středem 678 sekund
	na obou březích. Kapacita lodi je 40 míst. Celkově bylo převezeno 49 507 pasažérů. Maximální vytížení
	lodi dosáhlo 16 pasažérů v~jedné jízdě. Oproti Experimentu 2 je zřejmé, že se zvýšila využívanost lodi,
	ale stále nedosahuje maximální využití ani poloviny kapacity, pro kterou je loď konstruovaná.

	\subsubsection{Experiment 5}
	Tento experiment demonstruje sezónní provoz v~roce 2011, kdy počet převezených pasažérů
	za celý rok dosáhl rekordního čísla 323 240. Kapacita v~tomto experimentu je dána hodnotou 40 míst pro
	cestující. Generování pasažérů je dán exponenciálním rozložením se středem 46 sekund na obou
	březích. Doba sezóny je určena na 91 dní. V~tomto období bylo převezeno 242 025 pasažérů.
	Maximální využití lodi dosáhlo 39 obsazených míst. Dále je patrné, že žádný cestující
	nemusel čekat na další jízdu a všichni byli obslouženi. Maximální čas, kdy je pasažér v~systému je
	1250 sekund což je přibližně 21 minut.

	\subsection{Závěr plynoucí z~experimentů}
	Z~experimentů je patrné, že mimosezónní provoz přívozu je velmi řídce využívaný.
	Nejvhodnější řešení vidíme ve zkrácení doby provozu na 6 - 7 měsíců ročně např.
	od května do října, jelikož v~této době je počasí více příznivé a zájem cestujících
	je vyšší. Pokud by zůstal zachován celoroční provoz, tak uvažujeme za vhodné snížit
	frekvenci maximálně na 1 až 2 jízdy za hodinu, případně zkrátit pracovní dobu přívozu
	v~zimním období.

	Výsledky uvedených experimentů jsou dostupné v~souborech \texttt{experiment[1-5].out},
	které jsou vytvořeny po spuštění příkazu \texttt{make run}.

	\section{Závěr}
	Došli jsme k~závěru, že frekvence jízd mimo sezónu přívozu Podbaba - Podhoří je nevyhovující a je vhodné ji příslušným způsobem
	zredukovat a to z~důvodu minimálního využití. Tím je možno dosáhnout snížení opotřebení lodi a spotřeby paliva.
	Sezónní provoz je vhodně rozvržen, pokud je hlavním cílem navázat na hromadnou dopravu hl. města Prahy, jinak je možné
	tuto frekvenci snížit na 41 jízd za den.

	\newpage
	\section{Zdroje}
		\bibliography{documentation}


\end{document}
